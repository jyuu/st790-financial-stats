\documentclass[11pt,]{article}
\usepackage[left=1in,top=1in,right=1in,bottom=1in]{geometry}
\newcommand*{\authorfont}{\fontfamily{phv}\selectfont}
  \usepackage[]{mathpazo}
  
  
  \usepackage[T1]{fontenc}
\usepackage[utf8]{inputenc}



\usepackage{abstract}
\renewcommand{\abstractname}{}    % clear the title
\renewcommand{\absnamepos}{empty} % originally center

\renewenvironment{abstract}
{{%
  \setlength{\leftmargin}{0mm}
  \setlength{\rightmargin}{\leftmargin}%
}%
  \relax}
{\endlist}

\makeatletter
\def\@maketitle{%
  \newpage
  %  \null
  %  \vskip 2em%
    %  \begin{center}%
    \let \footnote \thanks
  {\fontsize{18}{20}\selectfont\raggedright  \setlength{\parindent}{0pt} \@title \par}%
}
%\fi
\makeatother


  
  
  \setcounter{secnumdepth}{0}

          
    
    \title{Homework 3  }
  
  
  
  \author{\Large Joyce Yu Cahoon\vspace{0.05in} \newline\normalsize\emph{}  }
  
  
  \date{}

\usepackage{titlesec}

\titleformat*{\section}{\normalsize\bfseries}
\titleformat*{\subsection}{\normalsize\itshape}
\titleformat*{\subsubsection}{\normalsize\itshape}
\titleformat*{\paragraph}{\normalsize\itshape}
\titleformat*{\subparagraph}{\normalsize\itshape}


  
      
  
  \newtheorem{hypothesis}{Hypothesis}
\usepackage{setspace}

\makeatletter
\@ifpackageloaded{hyperref}{}{%
  \ifxetex
  \PassOptionsToPackage{hyphens}{url}\usepackage[setpagesize=false, % page size defined by xetex
                                                 unicode=false, % unicode breaks when used with xetex
                                                 xetex]{hyperref}
  \else
    \PassOptionsToPackage{hyphens}{url}\usepackage[unicode=true]{hyperref}
  \fi
}

\@ifpackageloaded{color}{
  \PassOptionsToPackage{usenames,dvipsnames}{color}
}{%
  \usepackage[usenames,dvipsnames]{color}
}
\makeatother
\hypersetup{breaklinks=true,
bookmarks=true,
pdfauthor={Joyce Yu Cahoon ()},
pdfkeywords = {},  
pdftitle={Homework 3},
colorlinks=true,
citecolor=blue,
urlcolor=blue,
linkcolor=magenta,
pdfborder={0 0 0}}
\urlstyle{same}  % don't use monospace font for urls

% set default figure placement to htbp
\makeatletter
\def\fps@figure{htbp}
\makeatother

\setlength{\abovedisplayskip}{.2pt}
\setlength{\belowdisplayskip}{.2pt}
\usepackage{placeins}
\usepackage{setspace}
\usepackage{chngcntr}
\usepackage{multicol}
\usepackage{lscape}
\counterwithin{figure}{section}
\counterwithin{table}{section}
\usepackage{mathrsfs}
\usepackage{mathtools}
\usepackage{multirow}
\newtheorem{theorem}{Theorem}
\usepackage[linesnumbered,algoruled,boxed,lined,commentsnumbered]{algorithm2e}
\usepackage{bm}
\usepackage{framed}
\usepackage{xcolor}
\let\oldquote=\quote
\let\endoldquote=\endquote
\colorlet{shadecolor}{orange!15}
\renewenvironment{quote}{\begin{shaded*}\begin{oldquote}}{\end{oldquote}\end{shaded*}}
\newcommand{\V}[1]{{\bm{{#1}}}}


% add tightlist ----------
\providecommand{\tightlist}{%
\setlength{\itemsep}{0pt}\setlength{\parskip}{0pt}}

\begin{document}

% \pagenumbering{arabic}% resets `page` counter to 1 
%
% \maketitle

{% \usefont{T1}{pnc}{m}{n}
\setlength{\parindent}{0pt}
\thispagestyle{plain}
{\fontsize{18}{20}\selectfont\raggedright 
\maketitle  % title \par  

}

{
  \vskip 13.5pt\relax \normalsize\fontsize{11}{12} 
  \textbf{\authorfont Joyce Yu Cahoon} \hskip 15pt \emph{\small }   
  
}

}






\vskip 6.5pt


\noindent  \section{6.3}\label{section}

Let \(\V{Y}_t\) be a vector of excess returns of \(N\) assets. Consider
the multivariate linear regression model: \[
\V{Y}_t = \alpha + \beta Y_t^m + \V{\epsilon}_t
\] where \(\V{\epsilon}_t \sim N(0, \V{\Sigma})\) and
\(cov(Y_t^m, \epsilon) = 0\).

\begin{enumerate}
\def\labelenumi{\arabic{enumi}.}
\tightlist
\item
  Derive the MLE for \(\V{\alpha}\) and \(\V{\beta}\). You do not need
  to derive the MLE for \(\V{\Sigma}\) since this part is hard. You just
  take for granted that \(\V{\hat{\Sigma}}\) is the MLE.
\item
  Show that the MLRT for the null hypothesis \(H_0\): \(\V{\alpha} = 0\)
  is: \[
  T_2 = T[\log{(|\hat{\Sigma}_0|)} - \log{(\hat{\Sigma})}]
  \] where \(\V{\hat{\Sigma}}_0\) is the MLE under \(H_0\). Give the
  expression for \(\V{\hat{\Sigma}}_0\).
\end{enumerate}

\section{6.4}\label{section-1}

Consider the multifactor model: \[
\V{Y}_t = \V{\alpha} + \V{B}\V{X}_t + \V{\epsilon}_t
\] with observable factor \(\V{X}_t\) where
\(\mathbb{E}\V{\epsilon}_t =0\) and
\(cov(\V{X}_t, \V{\epsilon}_t) = 0\).

\begin{enumerate}
\def\labelenumi{\arabic{enumi}.}
\tightlist
\item
  Based on the 20 stock portfolios over a period of 60 months on the 3
  factors, it was computed that \(|\V{\hat{\Sigma}}_0| = 2.375\) and
  \(|\V{\hat{\Sigma}}| = 1.624\). Test if the multifactor model is
  consistent with the empirical data, \(H_0: \alpha = 0\).
\item
  Suppose that the beta's of the GE stock over the S\&P500 index
  (\(X_1\)), the size effect \(X_2\), and the book-to-market effect
  \(X_3\) are respectively 1.3, 0.3, -0.4. Assume further that over the
  last 10 years the average risk-free interest is 4\%, the average
  return of the S\&P500 is 11\%, the average difference of returns
  between the small large capitalization is 3\%, and the average
  difference of returns between the high and low book-to-market is 2\%,
  what is the expected return of the GE stock using the Fama-French
  model?
\end{enumerate}

\section{6.5}\label{section-2}

Consider the multi-factor model: \[
\V{Y}_t = \V{\alpha} + \V{B}\V{X}_t + \V{\epsilon}_t
\] with observable factor \(\V{X}_t\). 1. Suppose that the CAPM holds
and over the last five years, the average of the risk-free interest rate
is 3.5\% and the average return of the CRSP value-weighted index is
12.5\%. If the market \(\beta\) of a stock (with respect to the index)
is 1.3, what is the expected return of the stock? 2. Based on 15 stock
portfolios over a period of 60 months regressed on five factors without
knowing the risk-free interest rate, it is computed that
\(|\V{\hat{Sigma}}_0| = 2.425\) and \(|\V{\hat{\Sigma}}| = 1.742\). Test
if the multifactor model is consistent with the empirical data,
\(H_0: \alpha = 0\).\\
3. Suppose that the strict multi-factor model is correct so that
var(\(\V{\epsilon}\)) = \(\V{\Sigma}_0\) is a diagonal matrix and
\(\V{X}_t\) and \(\V{\epsilon}_t\) are uncorrelated. Show how to
estimate the covariance matrix of \(\V{Y}\) based on the past \(T\)
days' data: \[
\{(\V{X}_t, \V{Y}_t): t = 1, \ldots, T\}
\]

\section{6.8}\label{section-3}

Use the Fama-French 100 portfolios in the last five years to construct 3
common factors via the PCA based on the correlation matrix. Report the
variance explained by each principle components. Now, regress each of
the Fama-French 100 portfolios on these 3 principal components and
report the distribution (histogram) of the residual variances. Report
also the distribution of the variance of these 100 portfolios over the
same time period.
\newpage
\singlespacing 
\end{document}
