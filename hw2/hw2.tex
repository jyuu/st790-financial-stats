\documentclass[11pt,]{article}
\usepackage[left=1in,top=1in,right=1in,bottom=1in]{geometry}
\newcommand*{\authorfont}{\fontfamily{phv}\selectfont}
  \usepackage[]{mathpazo}
  
  
  \usepackage[T1]{fontenc}
\usepackage[utf8]{inputenc}



\usepackage{abstract}
\renewcommand{\abstractname}{}    % clear the title
\renewcommand{\absnamepos}{empty} % originally center

\renewenvironment{abstract}
{{%
  \setlength{\leftmargin}{0mm}
  \setlength{\rightmargin}{\leftmargin}%
}%
  \relax}
{\endlist}

\makeatletter
\def\@maketitle{%
  \newpage
  %  \null
  %  \vskip 2em%
    %  \begin{center}%
    \let \footnote \thanks
  {\fontsize{18}{20}\selectfont\raggedright  \setlength{\parindent}{0pt} \@title \par}%
}
%\fi
\makeatother


  
  
  \setcounter{secnumdepth}{0}

      \usepackage{color}
  \usepackage{fancyvrb}
  \newcommand{\VerbBar}{|}
  \newcommand{\VERB}{\Verb[commandchars=\\\{\}]}
  \DefineVerbatimEnvironment{Highlighting}{Verbatim}{commandchars=\\\{\}}
  % Add ',fontsize=\small' for more characters per line
  \usepackage{framed}
  \definecolor{shadecolor}{RGB}{248,248,248}
  \newenvironment{Shaded}{\begin{snugshade}}{\end{snugshade}}
  \newcommand{\AlertTok}[1]{\textcolor[rgb]{0.94,0.16,0.16}{#1}}
  \newcommand{\AnnotationTok}[1]{\textcolor[rgb]{0.56,0.35,0.01}{\textbf{\textit{#1}}}}
  \newcommand{\AttributeTok}[1]{\textcolor[rgb]{0.77,0.63,0.00}{#1}}
  \newcommand{\BaseNTok}[1]{\textcolor[rgb]{0.00,0.00,0.81}{#1}}
  \newcommand{\BuiltInTok}[1]{#1}
  \newcommand{\CharTok}[1]{\textcolor[rgb]{0.31,0.60,0.02}{#1}}
  \newcommand{\CommentTok}[1]{\textcolor[rgb]{0.56,0.35,0.01}{\textit{#1}}}
  \newcommand{\CommentVarTok}[1]{\textcolor[rgb]{0.56,0.35,0.01}{\textbf{\textit{#1}}}}
  \newcommand{\ConstantTok}[1]{\textcolor[rgb]{0.00,0.00,0.00}{#1}}
  \newcommand{\ControlFlowTok}[1]{\textcolor[rgb]{0.13,0.29,0.53}{\textbf{#1}}}
  \newcommand{\DataTypeTok}[1]{\textcolor[rgb]{0.13,0.29,0.53}{#1}}
  \newcommand{\DecValTok}[1]{\textcolor[rgb]{0.00,0.00,0.81}{#1}}
  \newcommand{\DocumentationTok}[1]{\textcolor[rgb]{0.56,0.35,0.01}{\textbf{\textit{#1}}}}
  \newcommand{\ErrorTok}[1]{\textcolor[rgb]{0.64,0.00,0.00}{\textbf{#1}}}
  \newcommand{\ExtensionTok}[1]{#1}
  \newcommand{\FloatTok}[1]{\textcolor[rgb]{0.00,0.00,0.81}{#1}}
  \newcommand{\FunctionTok}[1]{\textcolor[rgb]{0.00,0.00,0.00}{#1}}
  \newcommand{\ImportTok}[1]{#1}
  \newcommand{\InformationTok}[1]{\textcolor[rgb]{0.56,0.35,0.01}{\textbf{\textit{#1}}}}
  \newcommand{\KeywordTok}[1]{\textcolor[rgb]{0.13,0.29,0.53}{\textbf{#1}}}
  \newcommand{\NormalTok}[1]{#1}
  \newcommand{\OperatorTok}[1]{\textcolor[rgb]{0.81,0.36,0.00}{\textbf{#1}}}
  \newcommand{\OtherTok}[1]{\textcolor[rgb]{0.56,0.35,0.01}{#1}}
  \newcommand{\PreprocessorTok}[1]{\textcolor[rgb]{0.56,0.35,0.01}{\textit{#1}}}
  \newcommand{\RegionMarkerTok}[1]{#1}
  \newcommand{\SpecialCharTok}[1]{\textcolor[rgb]{0.00,0.00,0.00}{#1}}
  \newcommand{\SpecialStringTok}[1]{\textcolor[rgb]{0.31,0.60,0.02}{#1}}
  \newcommand{\StringTok}[1]{\textcolor[rgb]{0.31,0.60,0.02}{#1}}
  \newcommand{\VariableTok}[1]{\textcolor[rgb]{0.00,0.00,0.00}{#1}}
  \newcommand{\VerbatimStringTok}[1]{\textcolor[rgb]{0.31,0.60,0.02}{#1}}
  \newcommand{\WarningTok}[1]{\textcolor[rgb]{0.56,0.35,0.01}{\textbf{\textit{#1}}}}
        
    
    \title{Homework 2  }
  
  
  
  \author{\Large Joyce Yu Cahoon\vspace{0.05in} \newline\normalsize\emph{}  }
  
  
  \date{}

\usepackage{titlesec}

\titleformat*{\section}{\normalsize\bfseries}
\titleformat*{\subsection}{\normalsize\itshape}
\titleformat*{\subsubsection}{\normalsize\itshape}
\titleformat*{\paragraph}{\normalsize\itshape}
\titleformat*{\subparagraph}{\normalsize\itshape}


  
      
  
  \newtheorem{hypothesis}{Hypothesis}
\usepackage{setspace}

\makeatletter
\@ifpackageloaded{hyperref}{}{%
  \ifxetex
  \PassOptionsToPackage{hyphens}{url}\usepackage[setpagesize=false, % page size defined by xetex
                                                 unicode=false, % unicode breaks when used with xetex
                                                 xetex]{hyperref}
  \else
    \PassOptionsToPackage{hyphens}{url}\usepackage[unicode=true]{hyperref}
  \fi
}

\@ifpackageloaded{color}{
  \PassOptionsToPackage{usenames,dvipsnames}{color}
}{%
  \usepackage[usenames,dvipsnames]{color}
}
\makeatother
\hypersetup{breaklinks=true,
bookmarks=true,
pdfauthor={Joyce Yu Cahoon ()},
pdfkeywords = {},  
pdftitle={Homework 2},
colorlinks=true,
citecolor=blue,
urlcolor=blue,
linkcolor=magenta,
pdfborder={0 0 0}}
\urlstyle{same}  % don't use monospace font for urls

% set default figure placement to htbp
\makeatletter
\def\fps@figure{htbp}
\makeatother

\setlength{\abovedisplayskip}{.2pt}
\setlength{\belowdisplayskip}{.2pt}
\usepackage{placeins}
\usepackage{setspace}
\usepackage{chngcntr}
\usepackage{multicol}
\usepackage{lscape}
\counterwithin{figure}{section}
\counterwithin{table}{section}
\usepackage{mathrsfs}
\usepackage{mathtools}
\usepackage{multirow}
\newtheorem{theorem}{Theorem}
\usepackage[linesnumbered,algoruled,boxed,lined,commentsnumbered]{algorithm2e}
\usepackage{bm}
\usepackage{framed}
\usepackage{xcolor}
\let\oldquote=\quote
\let\endoldquote=\endquote
\colorlet{shadecolor}{orange!15}
\renewenvironment{quote}{\begin{shaded*}\quote}{\endquote\end{shaded*}}
\newcommand{\V}[1]{{\bm{\mathbf{\MakeLowercase{#1}}}}}


% add tightlist ----------
\providecommand{\tightlist}{%
\setlength{\itemsep}{0pt}\setlength{\parskip}{0pt}}

\begin{document}

% \pagenumbering{arabic}% resets `page` counter to 1 
%
% \maketitle

{% \usefont{T1}{pnc}{m}{n}
\setlength{\parindent}{0pt}
\thispagestyle{plain}
{\fontsize{18}{20}\selectfont\raggedright 
\maketitle  % title \par  

}

{
  \vskip 13.5pt\relax \normalsize\fontsize{11}{12} 
  \textbf{\authorfont Joyce Yu Cahoon} \hskip 15pt \emph{\small }   
  
}

}






\vskip 6.5pt


\noindent  \hypertarget{section}{%
\section{5.2}\label{section}}

Let \(s_A\) and \(s_B\) be the Sharpe ratio of portfolios A and B. Let
\(r_A\) and \(r_B\) be the expected returns of these two portfolios,
with standard deviation denoted by \(\sigma_A\) and \(\sigma_B\). Assume
that through self financing, portfolio A borrows
(\(\sigma_B/\sigma_A-1\)) at risk-free rate \(r_f\) to leverage so that
its risk is now the same as that of portfolio B.

Show that the excess return of leveraged investment in portfolio A is
larger than the expected return of portfolio B if \(s_A > s_B\). This
shows that the Sharpe ratio measures the efficiency of a portfolio.

\hypertarget{section-1}{%
\section{5.3}\label{section-1}}

Suppose that three mutual funds (conservative, growth and aggressive)
have annual log-returns of 15\%, 20\% and 30\% with volatility of 20\%,
30\% and 50\% respectively. The correlation between any of the 2 funds
is 0 and the risk-free rate is 5\%.

\begin{Shaded}
\begin{Highlighting}[]
\NormalTok{rf <-}\StringTok{ }\FloatTok{.05}
\NormalTok{vol <-}\StringTok{ }\KeywordTok{c}\NormalTok{(.}\DecValTok{20}\NormalTok{, }\FloatTok{.30}\NormalTok{, }\FloatTok{.50}\NormalTok{)}
\NormalTok{r <-}\StringTok{ }\KeywordTok{c}\NormalTok{(.}\DecValTok{15}\NormalTok{, }\FloatTok{.20}\NormalTok{, }\FloatTok{.30}\NormalTok{) }
\NormalTok{expected_return <-}\StringTok{ }\FloatTok{.15} 
\NormalTok{Y <-}\StringTok{ }\NormalTok{r }\OperatorTok{-}\StringTok{ }\NormalTok{rf }\CommentTok{# excess returns }
\NormalTok{gamma <-}\StringTok{ }\KeywordTok{diag}\NormalTok{(}\DataTypeTok{x =} \DecValTok{1}\NormalTok{, }\DataTypeTok{nrow =} \DecValTok{3}\NormalTok{) }
\NormalTok{vol <-}\StringTok{ }\KeywordTok{diag}\NormalTok{(vol, }\DataTypeTok{nrow =} \DecValTok{3}\NormalTok{) }
\NormalTok{Sigma <-}\StringTok{ }\NormalTok{vol }\OperatorTok\StringTok{ }\NormalTok{gamma }\OperatorTok\StringTok{ }\NormalTok{vol}
\NormalTok{partial_alpha <-}\StringTok{ }\KeywordTok{as.vector}\NormalTok{(}\KeywordTok{solve}\NormalTok{(Sigma) }\OperatorTok\StringTok{ }\NormalTok{Y ) }
\NormalTok{A <-}\StringTok{ }\KeywordTok{sum}\NormalTok{(partial_alpha }\OperatorTok{*}\StringTok{ }\NormalTok{Y)}\OperatorTok{/}\NormalTok{(expected_return }\OperatorTok{-}\StringTok{ }\NormalTok{rf)}
\end{Highlighting}
\end{Shaded}

\begin{enumerate}
\def\labelenumi{\arabic{enumi}.}
\item
  What is the min variance portfolio with these 3 mutual funds?
  \textgreater{} The min variance is given by
  \(\sigma^{*2} = \alpha^* \Sigma \alpha^*\)
\item
  Find the optimal portfolio allocation among the 3 mutual funds, if the
  expected return is set at 15\%. Give the associated standard deviation
  of this portfolio.
\item
  Compute the Sharpe ratio for the portfolio in A. How does it compare
  with that in B?
\end{enumerate}

\hypertarget{section-2}{%
\section{5.10}\label{section-2}}

Let \({\bm{\mathbf{\MakeLowercase{Y}}}}\) be the excess returns of risky
assets. Let
\(X = {\bm{\mathbf{\MakeLowercase{\alpha}}}}^T{\bm{\mathbf{\MakeLowercase{Y}}}}\)
be a portfolio with allocation vector
\({\bm{\mathbf{\MakeLowercase{\alpha}}}}\). Denote by
\(\Sigma = \text{var}({\bm{\mathbf{\MakeLowercase{Y}}}})\) and
\(\mu = \mathbb{E}({\bm{\mathbf{\MakeLowercase{Y}}}})\). Consider the
following decomposition:

\[
{\bm{\mathbf{\MakeLowercase{Y}}}} = \alpha + \beta X + \epsilon \quad \mathbb{E}(\epsilon) = 0 \quad \text{cov}(\epsilon, X) = 0
\] 1. Show that if
\({\bm{\mathbf{\MakeLowercase{\alpha}}}} = c\Sigma^{-1}\mu\) then
\(\alpha =0\). 2. Conversely if \(\alpha = 0\), there exists a constant
\(c\) such that
\({\bm{\mathbf{\MakeLowercase{\alpha}}}} = c\Sigma^{-1}\mu_0\)

\hypertarget{section-3}{%
\section{5.13}\label{section-3}}

Consider the following portfolio optimization problem with a risk-free
asset having return \(r_0\):

\[
\min{{\bm{\mathbf{\MakeLowercase{\alpha}}}}^T \Sigma {\bm{\mathbf{\MakeLowercase{\alpha}}}}} \quad \text{such that} \quad {\bm{\mathbf{\MakeLowercase{\alpha}}}}^T {\bm{\mathbf{\MakeLowercase{\mu}}}} + (1-{\bm{\mathbf{\MakeLowercase{\alpha}}}}^T{\bm{\mathbf{\MakeLowercase{1}}}}) r_0 = \mu
\] 1. The optimal solution is
\({\bm{\mathbf{\MakeLowercase{\alpha}}}} = P^{-1} (\mu-r_0)\Sigma^{-1}\mu_0\)
where
\(P = {\bm{\mathbf{\MakeLowercase{\mu}}}}_0^T \Sigma^{-1} {\bm{\mathbf{\MakeLowercase{\mu}}}}_0^T\)
is the squared Sharpe ratio, and
\({\bm{\mathbf{\MakeLowercase{\mu}}}}_0 = {\bm{\mathbf{\MakeLowercase{\mu}}}}-r_0{\bm{\mathbf{\MakeLowercase{1}}}}\)
is the vector of excess returns. 2. The variance of this portfolio is
\(\sigma^2 = (\mu - r_0)^2 / P\). 3. When \(r_0 < \mu\), show that
\(r_0 + P^{1/2} \sigma = \mu\), namely the optimal allocation for the
risky asset \({\bm{\mathbf{\MakeLowercase{\alpha}}}}\) is the tangent
portfolio.

\hypertarget{section-4}{%
\section{5.14}\label{section-4}}
\newpage
\singlespacing 
\end{document}
